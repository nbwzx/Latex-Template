\documentclass[12pt]{article}
\def\filedate{2021/2/23}
\usepackage{geometry}  %Set the size of each part of the page
\usepackage{fancyhdr}  %Set header, margin and footer
\usepackage{pifont}  %Special symbol font
\usepackage{graphicx}  %For illustration
\usepackage{ctex} %for Chinese
\usepackage{placeins} %\FloatBarries
\usepackage{extramarks}  %extra marks
\usepackage{mathrsfs} %Mathematical font
\usepackage{mathtools} %Mathematical tools
\usepackage{amsmath}  %Mathematical formula
\usepackage{amssymb}  %Mathematical fonts and symbols
\usepackage{amsthm} %Mathematical theorem format control
\usepackage{tikz} %Draw
\usepackage{algorithm} %algorithm
\usepackage{algpseudocode} %algpseudo code
\usepackage{tabularx} %Automatic calculation of column width
\usepackage{setspace} %Set space
\usepackage[framed,numbered,autolinebreaks,useliterate]{mcode} %Insert code
\usepackage{paralist} %Reduce the gap for itemize and enditemize
\newtheorem{lemma}{Lemma}
\newtheorem{theorem}{Theorem}
\newtheorem{remark}{Remark}
\newtheorem{corollary}{Corollary}
\renewcommand{\proofname}{\bf Proof}
\renewcommand{\thesection}{}
\let\itemize\compactitem 
\let\enditemize\endcompactitem 
\let\enumerate\compactenum 
\let\endenumerate\endcompactenum 
\let\description\compactdesc 
\let\enddescription\endcompactdesc
  \theoremstyle{plain}% default

%
% Basic Document Settings
%

\CTEXoptions[today=old]

\topmargin=-0.45in
\evensidemargin=0in
\oddsidemargin=0in
\textwidth=6.5in
\textheight=9.0in
\headsep=0.25in

\linespread{1.3}

\pagestyle{fancy}
\lhead{\hmwkAuthorName}
\chead{\hmwkClass\ : \hmwkTitle}
\rhead{\firstxmark}
\lfoot{\lastxmark}
\cfoot{\thepage}

\renewcommand\headrulewidth{0.4pt}
\renewcommand\footrulewidth{0.4pt}

\setlength\parindent{2em}

%
% Create Problem Sections
%

\newcommand{\enterProblemHeader}[1]{
    \nobreak\extramarks{}{Problem \arabic{#1} continued on next page\ldots}\nobreak{}
    \nobreak\extramarks{Problem \arabic{#1} (continued)}{Problem \arabic{#1} continued on next page\ldots}\nobreak{}
}

\newcommand{\exitProblemHeader}[1]{
    \nobreak\extramarks{Problem \arabic{#1} (continued)}{Problem \arabic{#1} continued on next page\ldots}\nobreak{}
    \stepcounter{#1}
    \nobreak\extramarks{Problem \arabic{#1}}{}\nobreak{}
}

\setcounter{secnumdepth}{0}
\newcounter{partCounter}
\newcounter{homeworkProblemCounter}
\setcounter{homeworkProblemCounter}{1}
\nobreak\extramarks{Problem \arabic{homeworkProblemCounter}}{}\nobreak{}

\newenvironment{homeworkProblem}{
    \section{Problem \arabic{homeworkProblemCounter}}
    \setcounter{section}{\arabic{homeworkProblemCounter}}
    \setcounter{partCounter}{1}
    \enterProblemHeader{homeworkProblemCounter}
}{
    \exitProblemHeader{homeworkProblemCounter}
}

%
% Homework Details
%   - Title
%   - Due date
%   - Class
%   - StudentID
%   - Instructor
%   - Author
%

\newcommand{\hmwkTitle}{Homework\ \#1}
\newcommand{\hmwkDueDate}{\today}
\newcommand{\hmwkClass}{Time Series Analysis}
\newcommand{\hmwkStudentID}{Student Number: 518070910121}
\newcommand{\hmwkClassInstructor}{Professor Shan Luo}
\newcommand{\hmwkAuthorName}{Zixing Wang}

%
% Title Page
%

\title{
    \vspace{0.5in}
    \textmd{\textbf{\hmwkClass: \hmwkTitle}}\\
    \normalsize\vspace{0.1in}\small{Finished\ on\ \hmwkDueDate\ }\\
   \vspace{0.1in}\large{\textit{\hmwkClassInstructor}}
    \vspace{3in}
\FloatBarrier
\begin{figure}[htb] 
\center{\includegraphics[width=5cm]  {DNA.png}} 
\end{figure}
 \FloatBarrier
 \author{\textbf{\hmwkAuthorName}}
\date{\hmwkStudentID}
}


%
% Various Helper Commands
%

% Alias for the Solution section header
\newcommand{\solution}{\par   \noindent   \textbf{Solution}\\ \indent}

% Probability commands: Expectation, Variance, Covariance, Bias and so on.
\newcommand{\E}{\mathrm{E}}
\newcommand{\Var}{\mathrm{Var}}
\newcommand{\Cov}{\mathrm{Cov}}
\newcommand{\Bias}{\mathrm{Bias}}
\newcommand{\alg}[1]{\textsc{\bfseries \footnotesize #1}}
\newcommand{\deriv}[1]{\frac{\mathrm{d}}{\mathrm{d}x} (#1)}
\newcommand{\pderiv}[2]{\frac{\partial}{\partial #1} (#2)}
\newcommand{\dx}{\mathrm{d}x}
\renewcommand{\part}[1]{\textbf{\large Part \Alph{partCounter}}\stepcounter{partCounter}\\}
\newcommand{\pic}[2]{
\FloatBarrier
\begin{figure}[htb] 
\center{\includegraphics[width=#2]  {#1}} 
\end{figure}
\FloatBarrier
}

\begin{document}
\lstset{numbers=left, xleftmargin=1.5em,xrightmargin=0.5em, aboveskip=0.5em}
\maketitle
\pagebreak
 
\begin{homeworkProblem}
1.3 附录B中的 B6 是 1973 至 1978 年美国在意外事故中的死亡人数.
\FloatBarrier
\renewcommand\arraystretch{1.5}
\begin{table}[H]
\begin{tabular}{|c|c|c|c|c|c|c|c|c|c|c|c|c|}
\hline
年份   & 1月   & 2月   & 3月   & 4月   & 5月    & 6月    & 7月    & 8月    & 9月   & 10月  & 11月  & 12月  \\ \hline
1973 & 9007 & 8106 & 8928 & 9137 & 10017 & 10826 & 11317 & 10744 & 9713 & 9938 & 9161 & 8927 \\ \hline
1974 & 7750 & 6981 & 8038 & 8422 & 8714  & 9512  & 10120 & 9823  & 8743 & 9192 & 8710 & 8680 \\ \hline
1975 & 8162 & 7306 & 8124 & 7870 & 9387  & 9556  & 10093 & 9620  & 8285 & 8433 & 8160 & 8034 \\ \hline
1976 & 7717 & 7461 & 7776 & 7925 & 8634  & 8945  & 10078 & 9179  & 8037 & 8488 & 7874 & 8647 \\ \hline
1977 & 7792 & 6957 & 7726 & 8106 & 8890  & 9299  & 10625 & 9302  & 8314 & 8850 & 8265 & 8796 \\ \hline
1978 & 7836 & 6892 & 7791 & 8129 & 9115  & 9434  & 10484 & 9827  & 9110 & 9070 & 8633 & 9240 \\ \hline
\end{tabular}
\end{table}
\FloatBarrier
利用至少两种方法对该时间序列进行分解.要求如下:
\begin{itemize}
\item[(1)] 画出数据图,给出数据的周期 $T$;
\item[(2)] 给出趋势项、周期项和随机项的计算公式;
\item[(3)] 画出趋势项、周期项和随机项的数据图;
\item[(4)] 对 1979 年的意外死亡人数做出预测.
\end{itemize}
\solution
首先介绍一个引理。
\begin{lemma}[Size Of Left Coset]
Let $H$ be a finite subgroup of a group $G$.  Then each left
coset of $H$ in $G$ has the same number of elements as $H$.
\end{lemma}

\begin{theorem}[Lagrange's Theorem]
Let $G$ be a finite group, and let $H$ be a subgroup
of $G$.  Then the order of $H$ divides the order of $G$.
\end{theorem}

\begin{proof}
Let $z$ be some element of $xH \cap yH$.  Then $z = xa$ for some $a \in H$, and $z = yb$ for some $b \in H$. If $h$ is any element of $H$ then $ah \in H$ and $a^{-1}h \in H$, since $H$ is a subgroup of $G$. But $zh = x(ah)$ and $xh = z(a^{-1}h)$ for all $h \in H$. Therefore $zH \subset xH$ and $xH \subset zH$, and thus $xH = zH$.  Similarly $yH = zH$, and thus $xH = yH$, as required.
\end{proof}

\lstset{numbers=left, xleftmargin=1.5em,xrightmargin=0.5em, aboveskip=0.5em}
\begin{lstlisting}[language=R]
death.month<- array(0, c(72))
death.detrended<-death.people-death.trended
ymat <- matrix(c(death.detrended), byrow=TRUE, ncol=12)
death.avg <- apply(ymat, 2, mean, na.rm=TRUE)
for  (i in 1:72) {
death.month[i]<-death.avg[(i-1)%%12+1]
}
\end{lstlisting}

\pic{fig1.png}{10cm}

%\lstinputlisting[language=Python]{D:/python/program/active.py}

对于房价的预测而言,url, id, Cid, DOM, followers是没有用的信息。对上述量进行删除处理。
\lstset{numbers=left, xleftmargin=1.5em,xrightmargin=0.5em, aboveskip=0.5em}
\begin{lstlisting}[language=python]
df = df.drop(['url', 'id', 'Cid', 'DOM', 'followers'], axis = 1)
\end{lstlisting}
\end{homeworkProblem}
\pagebreak
  
\begin{homeworkProblem}
5.38. Show that if $X$ and $Y$ are regular, then so is the product space $X \times Y .$ Conclude that $\mathbb{R}^{n}$ is regular.
\solution
1. For any point $(x, y) \in X \times Y$, it's the product of two closed sets in $X$ and $Y$, so it's closed. 

2. For any point $(x, y) \in X \times Y$ and any closed set $C=U \times V,$ where $U$ is closed in $X$ and $V$ is closed in $Y$ and $x \notin U$ and $y \notin V$. $X$ and $Y$ are regular, so for $x$ and $U,$ we have there exists open sets $A_{1}$ and $A_{2}$ such that $x \in A_{1}$ and $U \subset A_{2}$ and $A_{1} \cap A_{2}=\varnothing$. Similarly, for $y$ and $V$, we have there exists open sets $B_{1}$ and $B_{2}$ such that $y \in B_{1}$ and $V \subset B_{2}$ and $B_{1} \cap B_{2}=\varnothing$.
Then we know that $(x, y) \in A_{1} \times B_{1}$ and $C \subset A_{2} \times B_{2},$ where $A_{1} \times B_{1}$ and $A_{2} \times B_{2}$ are open, and $\left(A_{1} \times B_{1}\right) \cap\left(A_{2} \times B_{2}\right)=\varnothing$.

$\mathbb{R}$ is regular, so use this conclusion n times, we know that $\mathbb{R}^{n}$ is regular.
\end{homeworkProblem}
\pagebreak

\end{document}
