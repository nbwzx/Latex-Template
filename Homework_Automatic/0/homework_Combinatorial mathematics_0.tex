\documentclass[12pt]{article}
\def\filedate{2021/9/21}
\def\updatewebsite{https://github.com/nbwzx/Latex-Template}
\input setting.tex

\begin{document}
\lstset{numbers=left, xleftmargin=1.5em,xrightmargin=0.5em, aboveskip=0.5em}
\maketitle
\pagebreak
\begin{homeworkProblem}
1.3 附录B中的 B6 是 1973 至 1978 年美国在意外事故中的死亡人数.
\FloatBarrier
\renewcommand\arraystretch{1.5}
\begin{table}[H]
\begin{tabular}{|c|c|c|c|c|c|c|c|c|c|c|c|c|}
\hline
年份   & 1月   & 2月   & 3月   & 4月   & 5月    & 6月    & 7月    & 8月    & 9月   & 10月  & 11月  & 12月  \\ \hline
1973 & 9007 & 8106 & 8928 & 9137 & 10017 & 10826 & 11317 & 10744 & 9713 & 9938 & 9161 & 8927 \\ \hline
1974 & 7750 & 6981 & 8038 & 8422 & 8714  & 9512  & 10120 & 9823  & 8743 & 9192 & 8710 & 8680 \\ \hline
1975 & 8162 & 7306 & 8124 & 7870 & 9387  & 9556  & 10093 & 9620  & 8285 & 8433 & 8160 & 8034 \\ \hline
1976 & 7717 & 7461 & 7776 & 7925 & 8634  & 8945  & 10078 & 9179  & 8037 & 8488 & 7874 & 8647 \\ \hline
1977 & 7792 & 6957 & 7726 & 8106 & 8890  & 9299  & 10625 & 9302  & 8314 & 8850 & 8265 & 8796 \\ \hline
1978 & 7836 & 6892 & 7791 & 8129 & 9115  & 9434  & 10484 & 9827  & 9110 & 9070 & 8633 & 9240 \\ \hline
\end{tabular}
\end{table}
\FloatBarrier
利用至少两种方法对该时间序列进行分解.要求如下:
\begin{itemize}
\item[(1)] 画出数据图,给出数据的周期 $T$;
\item[(2)] 给出趋势项、周期项和随机项的计算公式;
\item[(3)] 画出趋势项、周期项和随机项的数据图;
\item[(4)] 对 1979 年的意外死亡人数做出预测.
\end{itemize}
\solution
首先介绍一个引理。
\begin{lemma}[Size Of Left Coset]
Let $H$ be a finite subgroup of a group $G$.  Then each left
coset of $H$ in $G$ has the same number of elements as $H$.
\end{lemma}

\begin{theorem}[Lagrange's Theorem]
Let $G$ be a finite group, and let $H$ be a subgroup
of $G$.  Then the order of $H$ divides the order of $G$.
\end{theorem}

\begin{proof}
Let $z$ be some element of $xH \cap yH$.  Then $z = xa$ for some $a \in H$, and $z = yb$ for some $b \in H$. If $h$ is any element of $H$ then $ah \in H$ and $a^{-1}h \in H$, since $H$ is a subgroup of $G$. But $zh = x(ah)$ and $xh = z(a^{-1}h)$ for all $h \in H$. Therefore $zH \subset xH$ and $xH \subset zH$, and thus $xH = zH$.  Similarly $yH = zH$, and thus $xH = yH$, as required.
\end{proof}

\lstset{numbers=left, xleftmargin=1.5em,xrightmargin=0.5em, aboveskip=0.5em}
\begin{lstlisting}[language=R]
death.month<- array(0, c(72))
death.detrended<-death.people-death.trended
ymat <- matrix(c(death.detrended), byrow=TRUE, ncol=12)
death.avg <- apply(ymat, 2, mean, na.rm=TRUE)
for  (i in 1:72) {
death.month[i]<-death.avg[(i-1)%%12+1]
}
\end{lstlisting}
\pic{fig1.png}{0.8}{Abstract algebra}{fig1}
%pic{file path and file name}{width(xcm or x( means x\textwidth)}{caption(optional)}{label(optional)}

对于房价的预测而言,url, id, Cid, DOM, followers是没有用的信息。对上述量进行删除处理。
\lstset{numbers=left, xleftmargin=1.5em,xrightmargin=0.5em, aboveskip=0.5em}
\begin{lstlisting}[language=python]
df = df.drop(['url', 'id', 'Cid', 'DOM', 'followers'], axis = 1)
\end{lstlisting}

%\lstinputlisting[language=Python]{D:/python/program/active.py}
\end{homeworkProblem}
\pagebreak
  
\begin{homeworkProblem}
5.38. Show that if $X$ and $Y$ are regular, then so is the product space $X \times Y .$ Conclude that $\mathbb{R}^{n}$ is regular.
\solution
1. For any point $(x, y) \in X \times Y$, it's the product of two closed sets in $X$ and $Y$, so it's closed. 

2. For any point $(x, y) \in X \times Y$ and any closed set $C=U \times V,$ where $U$ is closed in $X$ and $V$ is closed in $Y$ and $x \notin U$ and $y \notin V$. $X$ and $Y$ are regular, so for $x$ and $U,$ we have there exists open sets $A_{1}$ and $A_{2}$ such that $x \in A_{1}$ and $U \subset A_{2}$ and $A_{1} \cap A_{2}=\varnothing$. Similarly, for $y$ and $V$, we have there exists open sets $B_{1}$ and $B_{2}$ such that $y \in B_{1}$ and $V \subset B_{2}$ and $B_{1} \cap B_{2}=\varnothing$.
Then we know that $(x, y) \in A_{1} \times B_{1}$ and $C \subset A_{2} \times B_{2},$ where $A_{1} \times B_{1}$ and $A_{2} \times B_{2}$ are open, and $\left(A_{1} \times B_{1}\right) \cap\left(A_{2} \times B_{2}\right)=\varnothing$.

$\mathbb{R}$ is regular, so use this conclusion n times, we know that $\mathbb{R}^{n}$ is regular.
\end{homeworkProblem}
\pagebreak

\end{document}

